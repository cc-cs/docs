\input{myarticlepreamble.tex}
\input{cccs.tex}
\renewcommand\TITLE{CCCS: Spring 2016 Agenda}

\begin{document}
\topmatter


\section{PIZZA}
\label{sec-1}

As Ryan would have it \ldots{}


\newpage


\section{Introductions}
\label{sec-2}

Welcome to the Computer Science club kickoff. There's one every semester
and the idea is to have an informal first meeting to orient everyone towards
the club's plans for the semester. This is especially important for new
students.

\subsection{Google Group}
\label{sec-2-1}

The CS department has a Google Group that is used as our mailing list. The
group is cc\_cs and the mailing address for the group is cc\_cs@googlegroups.com.
In particular, it is used for department-wide announcments that include events
and employment opportunities. If you aren't in the group currently, please fill
out the form being passed around with your email.

\subsection{Calendar}
\label{sec-2-2}

There's a calendar of events that you can access through the frontpage at
\url{http://bit.ly/yliow0}.

\subsection{Icebreaker}
\label{sec-2-3}

Let's go around and have everyone introduce themselves. Include your interests
in and out of CS and some interesting facts about yourself.


\newpage


\section{Columbia College Computer Science Club}
\label{sec-3}

\subsection{Information}
\label{sec-3-1}

\begin{description}
\item[{Group Email:}] cc.cs.officers@googlegroups.com
\item[{Advisor:}] Dr. Yihsiang Liow (yliow@ccis.edu)
\item[{Officers:}] \hspace*{1em}
\begin{description}
\item[{President:}] Ryan Frappier (rbfrappier1@cougars.ccis.edu)
\item[{Vice-president:}] Faizal Glenn (feglenn1@couars.ccis.edu)
\item[{Treasurer:}] Andrew Woods (amwoods3@cougars.ccis.edu)
\item[{Secretary:}] Rotshak Dakup (rjdakup1@cougars.ccis.edu)
\item Chris Heskett (cjheskett1@cougars.ccis.edu)
\item Garrett Waage (gjwaage1@cougars.ccis.edu)
\item Michael \textquoteleft{}Mikey\textquoteright{} Thomas (mgthomas2@cougars.ccis.edu)
\item Nathan Haggerman (nmhaggerman1@cougars.ccis.edu)
\item Ujjwal Pandey (ujpandey1@cougars.ccis.edu)
\end{description}
\item[{Locations:}] \hspace*{1em}
\begin{description}
\item[{STC 321:}] Computer Science \textquoteleft{}Linux\textquoteright{} Lab
\item[{STC 223:}] Dr. Liow's Office
\item[{BUH 104 / BUH 107:}] Buchanan lab classrooms
\end{description}
\end{description}

\newpage

\subsection{Computer Science Hangouts}
\label{sec-3-2}

We meet every day from 3 to 5 pm in one of the Buchanan labs.

There are seniors and Dr. Liow in attendance who will help with anything CS
related. Everyone's encouraged to help each other where possible as well.

It's fine to come just to hang out if no help is needed for classwork. It's
also fine to come discuss CS related topics outside of classwork.

The exact room schedule is posted on the calendar, but we are in BUH104 on
Mondays, Wednesdays and Fridays and in BUH107 on Tuesdays and Thursdays. There
are a couple of days where both those rooms were occupied for something else
and we will be in some other BUH room for those days. Just look around the
Buchanan rooms between 3 to 5 pm and you will find us in one of them.

\begin{description}
\item[{WHAT:}] Computer Science Hangouts
\item[{WHY:}] Hangout with others in the department and get help if you need any
\item[{WHEN:}] 3pm to 5 pm, every weekday
\item[{WHERE:}] Buchanan Hall, BUH104 or BUH107, just check the rooms
\end{description}

\newpage

\subsection{Club Meetings}
\label{sec-3-3}

The club will have meetings on demand. We have events planned throughout the
semester that we will send periodic information about as we approach them
through the Google Group and other channels. We also treat Computer Science
Hangout sessions every Friday from 3 to 5 pm in BUH104 as informal club
meetings. Club members should make every attempt to join the Friday hangout
sessions if possible.

Apart from the regular club things you can expect of a typical
college club, you can expect to learn valuable information through occasional
events such as:
\begin{itemize}
\item workshops on various tools that a programmer should be aware of
\item talks on career planning, searching for internships and jobs, interview
preparation and so on
\item forums on various special interest topics such as game development,
artificial intelligence and more
\item presentations from students who want to show off their work
\end{itemize}

If there's an event such as these scheduled, we will notify the group and the
club will have an official meeting. Most of these will be scheduled during the
Friday hangout sessions but we will consider organizing events with heavy
demand at other times as well.

We will typically ask for volunteers to produce content for and/or run the next
meeting during a meeting and decide among the options. Everyone's encouraged to
chip in. We would love for new students in particular to get involved and
produce some content for a meeting, or even take charge of some meetings
altogether.

\textbf{Note} that even if there's no event happening, we will still meet for the
regular Computer Science Hangout session and the session will be treated as an
informal club meeting.

\newpage

\subsection{Other events}
\label{sec-3-4}

We will also have other fun, social events outside of the weekly informal
meetings and formal events described above. There are ideas flying around that
include but aren't limited to the following:
\begin{itemize}
\item Movie nights
\item Jam sessions
\item Hackathons
\item Game days
\item Fun(d) raising events
\end{itemize}

See further ahead for details on specific instances of some of these.

Any other ideas are welcome. We also need volunteers to help us organize these
events. If you fancy any of these, or if you have a different event you would
like to lead the club into, get in touch with one of the club officers.


\newpage


\section{Linux Workshop}
\label{sec-4}

Next Friday, January 29, Ujjwal will be running a linux workshop during the CS club
meeting, i.e., 3 to 5 pm in BUH104.

We will cover basic Linux commands and demonstrate a workflow to write C++
programs that makes working on Dr. Liow's assignments much more fun than
working in Visual Studio does.

This is recommended for any CISS245 and above student who doesn't yet know of
Linux. Interested students from lower classes are nonetheless very welcome.

There will be further Linux workshops covering other workflows and more advanced
topics later on. Especially if there is demand.

\begin{description}
\item[{WHAT:}] Linux Workshop
\item[{WHY:}] Learn a new Operating System that is much more streamlined for programming
\item[{WHEN:}] 3 pm to 5 pm, January 29
\item[{WHERE:}] BUH104
\end{description}


\newpage


\section{MidwayUSA Visit}
\label{sec-5}

Our department has a strong relationship with the local companies of note. One
of the biggest local companies, MidwayUSA, has a large number of our graduates
working for them. They have invited us to tour their company. We will visit
them either on February 4 or February 5. 

We will decide the time and date based on how many of us here can make
it. Please indicate on the form being passed around what time you would prefer.

\begin{description}
\item[{WHAT:}] MidwayUSA Visit
\item[{WHY:}] Watch professional programmers in action and start building a network
\item[{WHEN:}] February 4 or February 5, exact time to be decided
\item[{WHERE:}] MidwayUSA, transportation will be provided
\end{description}


\newpage


\section{Movie Night}
\label{sec-6}

We will run a movie night on friday, February 19. The location isn't yet
finalized, and we will send information out in the next week or so.

We will watch one of the all-time classics: Yojimbo by Akira Kurosawa featuring
Toshiro Mifune. The movie is in Japanese so it will be somewhat of a cultural
experience as well. Among other things in it's legacy, Yojimbo was the primary
inspiration for other great movies such as \textquotedblleft{}The Good, the Bad and the
Ugly\textquotedblright{}.

\begin{description}
\item[{WHAT:}] Movie Night featuring Yojimbo by Akira Kurosawa
\item[{WHY:}] Watch a classic movie featuring a renowned actor-director combination
\item[{WHEN:}] 7 pm to 10 pm on February 19
\item[{WHERE:}] To Be Decided
\end{description}


\newpage


\section{Computer Science Jam Session (CS.JS)}
\label{sec-7}

Our annual musical event makes a return this year as well. On the ides of
March, we will meet at the Dorsey Chapel between 11:00 and 12:30 and show off
our musical abilities and masterpieces. Everyone is encouraged to do something,
even if it's just miming along their favorite song's music video.

\begin{description}
\item[{WHAT:}] Computer Science Jam Session (CS.JS)
\item[{WHY:}] Share your music with others and listen to what others are listening to!
\item[{WHEN:}] 11 am to 12:30 pm on March 15
\item[{WHERE:}] Dorsey Chapel
\end{description}


\newpage


\section{Hackathon}
\label{sec-8}

Propose some ideas that you want to hack some code for. Pick the idea that
interests you the most. Hack away. For a day. Internal only (i.e., CC students
only) for now. Tentative date is March 19.

\begin{description}
\item[{WHAT:}] CC Hackathon
\item[{WHY:}] Share your awesome application idea and hack a prototype in a day
\item[{WHEN:}] 8 am to 9 pm on March 19
\item[{WHERE:}] To be decided
\end{description}


\newpage


\section{Portal into Computer Science}
\label{sec-9}

Last year, we organized an event we called Portal Into Computer Science (PiCS)
for high school students in the area. We had a trivia challenge focused on CS
and a programming contest for the students with minor prizes. It was a great
success and we are going to do it again this year. We will probably run it
around April. We need volunteers for the event planning and management. Reach
out to one of the club officers if you are interested in being a part of this.

\begin{description}
\item[{WHAT:}] Portal Into Computer Science (PiCS)
\item[{WHY:}] Help motivate the next generation of Computer Science students and have some fun
\item[{WHEN:}] Around April, exact time and date to be decided
\item[{WHERE:}] To Be Decided
\end{description}


\newpage


\section{CS T-shirt}
\label{sec-10}

As a tradition, we order t-shirts associated with the club every year. We offer
a very good deal. The price will be around 12 \$ for a short-sleeved t-shirt
based on previous years. The proceeds from the sale are used to fund club
meetings. This year we will be using a new design that ties the t-shirt with
the club even stronger. More details will follow. We would like to have a rough
headcount, so please note in the form being passed around if you are interested
in buying a CS t-shirt. We will probably be collecting money for this sometime
towards the end of February.

Faizal has some information on the new design to share with us.

\begin{description}
\item[{WHAT:}] CS T-shirts
\item[{WHY:}] Wear the club with pride and have a souvenir for later years
\item[{WHEN:}] Late February, early March
\item[{PRICE}] To Be Decided (\textasciitilde{}\$12)
\end{description}


\newpage


\section{Fund Raising}
\label{sec-11}

Ideas and volunteers welcome. Make it fun if possible.

\end{document}
